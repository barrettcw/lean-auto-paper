\section{Introduction}

  Interactive Theorem Provers (ITPs) \cite{Harrison2014HistoryOI}
  are widely used in formal mathematics and software/hardware verification. Hammers
  \cite{Blanchette2016HammeringTQ,Czajka2018HammerFC} are proof automation tools for
  ITPs which utilize Automated Theorem Provers (ATPs, including Satisfiability Modulo Theories (SMT) solvers).
  When using ITPs, straightforward but tedious proof
  tasks often arise during the proof development process.  Due to the limited
  built-in automation in ITPs, discharging these can require significant
  manual effort. Hammers have proved useful
  because they can solve many of these proof tasks automatically \cite{Paulson2012ThreeYO}.
  
  A hammer has three main components: premise selection, translation from ITP to
  ATP, and proof reconstruction from ATP to ITP. Premise selection collects
  the necessary information needed to solve a proof task, translation exports
  the collected information from the ITP to the ATP, and proof reconstruction generates
  a proof in the ITP based on the output of the ATP. The discrepancies between logical systems of ATPs and ITPs pose
  significant challenges to translation procedures between them.
  Several popular ITPs are based on highly expressive logical systems.
  For example, Isabelle \cite{Isabelle} is based on polymorphic higher-order logic, while
  Coq \cite{CoqRefMan}, Lean4 \cite{Lean4} and Agda \cite{Agda}
  are based on an even more expressive logical system called dependent type
  theory \cite{LambdaWithType,Coquand1988}.\footnote{Or \textit{calculus of inductive constructions (CIC)}, depending
  on whether inductive types are considered as an extension.}
  Moreover, features such as typeclasses \cite{TypeClassHaskell}, universe polymorphism \cite{UPolyCoq} and inductive types \cite{CICIndDef}
  are commonly used as extensions to the base logical system to enhance usability of the ITPs.
  On the other hand, ATPs are usually based on less expressive logical systems such
  as first-order logic (FOL) \cite{CVC5,Vampire,Z3Paper,EProver} and (in recent years)
  higher-order logic (HOL) \cite{HOVampire,ZipperpositionMakeWork,HOEProver}.
  An overview of the various logical systems relevant to our work will be given in Sect. \ref{sublogsys}.

  The focus of our paper is the Lean4 to ATP translation procedure in Lean-auto.
  We note that Lean-auto does have a proof reconstruction procedure which fully supports
  one of the three types of ATPs we use to evaluate Lean-auto.
  Ongoing projects are expected to implement premise selection and more proof reconstruction support.
  See Sect. \ref{sectexpr} for more discussion.

  There are two existing approaches for translation from more expressive
  logical systems to less expressive ones: encoding-based translation and monomorphization.
  Encoding-based translation is used in CoqHammer \cite{Czajka2018HammerFC}
  to translate Coq into untyped FOL. Monomorphization is used to
  eliminate polymorphism in Isabelle Sledgehammer \cite{Blanchette2016HammeringTQ,MonoPaper,Paulson2012ThreeYO}.
  % TODO: Our experiments suggest
  We found that encoding-based translation tends to produce large outputs
  on complex problems arising from real ITP use cases, which negatively affects the performance of ATPs.
  Therefore, we decided to use monomorphization in Lean-auto. An overview of these two
  translation methods and related discussion is covered in Sect. \ref{subencmon}.

  Since ATPs have started supporting HOL in recent years \cite{HOVampire,ZipperpositionMakeWork,HOEProver},
  we decided that Lean-auto should translate Lean4 to HOL. The overall translation has
  two stages: preprocessing (Sect. \ref{sectprep}) and monomorphization.
  Monomorphization itself has three stages: quantifier instantiation (Sect. \ref{sectinst}),
  $\lambda_\to^*$ abstraction (Sect. \ref{sectabst}), and universe lifting (Appendix \ref{appulift}).
  Roughly speaking, preprocessing translates Lean4 into dependent type
  theory,\footnote{As mentioned before, Lean4 is different from
  dependent type theory because it also has various language features.}
  and monomorphization translates dependent type theory into HOL.
  The monomorphization procedure is inspired by Isabelle Sledgehammer.
  However, since dependent type theory is considerably different from
  Isabelle's HOL, the monomorphization procedure is thoroughly redesigned,
  and presented in a different way in our paper. Challanges related to
  dependent type theory and Lean4 are discussed in Sect. \ref{exqdet}.
