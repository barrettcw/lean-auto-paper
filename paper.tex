% This is samplepaper.tex, a sample chapter demonstrating the
% LLNCS macro package for Springer Computer Science proceedings;
% Version 2.21 of 2022/01/12
%
\documentclass[runningheads]{llncs}
%
\usepackage[T1]{fontenc}
% T1 fonts will be used to generate the final print and online PDFs,
% so please use T1 fonts in your manuscript whenever possible.
% Other font encondings may result in incorrect characters.
%
\usepackage{graphicx}
% Used for displaying a sample figure. If possible, figure files should
% be included in EPS format.
%
% If you use the hyperref package, please uncomment the following two lines
% to display URLs in blue roman font according to Springer's eBook style:
%\usepackage{color}
%\renewcommand\UrlFont{\color{blue}\rmfamily}
%\urlstyle{rm}
%
% Packages required for listing Lean code
\usepackage[T1]{fontenc}
\usepackage[utf8]{inputenc}
\usepackage{listings}
\usepackage{color}
\def\lstlanguagefiles{lstlean.tex}
\definecolor{hypcolor}{rgb}{0.85, 0.48, 0.0}
\lstdefinestyle{leanHH}{
  language=lean,
  moredelim=**[is][\color{hypcolor}]{|!}{!|},
}
\usepackage{amssymb}
\newsavebox\vEq
\newsavebox\vmap
\newsavebox\vrev
\newsavebox\vmaprev
\newsavebox\vrevrev
%\definecolor{keywordcolor}{rgb}{0.7, 0.1, 0.1}   % red
\definecolor{keywordcolor}{rgb}{0.0, 0.0, 0.0}   % black
\definecolor{tacticcolor}{rgb}{0.0, 0.3, 0.8}    % blue
\definecolor{commentcolor}{rgb}{0.4, 0.4, 0.4}   % grey
%\definecolor{symbolcolor}{rgb}{0.0, 0.1, 0.6}    % blue
\definecolor{symbolcolor}{rgb}{0.0, 0.0, 0.0}    % black
%\definecolor{sortcolor}{rgb}{0.1, 0.5, 0.1}      % green
\definecolor{sortcolor}{rgb}{0.0, 0.0, 0.0}      % black
\definecolor{attributecolor}{rgb}{0.7, 0.1, 0.1} % red

\usepackage{fancybox}
\makeatletter
\newenvironment{CenteredBox}{% 
\begin{Sbox}}{% Save the content in a box
\end{Sbox}\centerline{\parbox{\wd\@Sbox}{\TheSbox}}}% And output it centered
\makeatother

\usepackage{amsmath}
\usepackage{algorithm2e}
\SetKwProg{Fn}{Function}{}{end}
\SetKwSwitch{Switch}{Case}{Other}{match}{}{case}{otherwise}{}{}
\SetKwFor{For}{for}{do}{}
\SetKwInOut{Input}{In}
\SetKwInOut{Output}{Out}
\SetKw{Continue}{continue}
\SetKw{Break}{break}

\begin{document}

\title{Lean-auto: An Interface between Lean4 and Automated Theorem Provers}
%
%\titlerunning{Abbreviated paper title}
% If the paper title is too long for the running head, you can set
% an abbreviated paper title here
%
\author{
  Yicheng Qian\inst{3}{ % TODO: \orcidID{}
  } \and
  Joshua Clune\inst{1}\orcidID{0000-0003-4047-6196} \and
  Jasmin Blanchette\inst{2}\orcidID{0000-0002-8367-0936} \and
  Clark Barrett\inst{3}\orcidID{0000-0002-9522-3084} \and
  Jeremy Avigad\inst{1}\orcidID{0000-0003-1275-315\textrm{X}} \\
  % TODO: To be determined
  % TODO: Please ask Jasmin
}
%
\authorrunning{Y. Qian et al.}
% First names are abbreviated in the running head.
% If there are more than two authors, 'et al.' is used.
%
\institute{
  Carnegie Mellon University, Pittsburgh, PA 15213, USA \and
  Heinrich-Heine-Universität Düsseldorf, 40225 Düsseldorf, Germany \and
  Stanford University, Stanford, CA 94305, USA
}
%
\maketitle              % typeset the header of the contribution
\begin{abstract}
  \textbf{TODO}
  
  \keywords{Proof Automation \and Lean \and Dependent Type Theory}
\end{abstract}
\section{Introduction}

  Interactive Theorem Provers (ITPs)\cite{Harrison2014HistoryOI}
  are widely used in formal mathematics and software/hardware verification. Hammers
  \cite{Blanchette2016HammeringTQ}\cite{Czajka2018HammerFC}, a type of proof automation tool for
  ITPs, utilize Automated Theorem Provers (ATPs, including SMT solvers) to automatically solve proof goals
  arising from the formalization process. A hammer has three main components:
  premise selection, translation from the ITP's logical system to the ATP's
  logical system, and proof reconstruction from the ATP's logical system to
  the ITP's logical system. Several popular ITPs, including
  Coq\cite{CoqRefMan}, Lean\cite{Lean4} and Agda\cite{Agda},
  are based on a highly expressive logical system called dependent type theory \cite{Coquand1988}.
  On the other hand, ATPs are usually based on less expressive logical systems such
  as first-order logic (FOL) and higher-order logic (HOL). This discrepancy in
  logical systems poses a significant challenge to the translation procedure from
  ITPs to ATPs.

  There are two existing approaches for translation from more expressive
  logical systems to less expressive ones, namely encoding-based translation and monomorphization.
  Encoding-based translation is used in CoqHammer\cite{Czajka2018HammerFC}
  to translate Coq into untyped FOL, while monomorphization is used to
  eliminate polymorphism in Isabelle Sledgehammer\cite{Blanchette2016HammeringTQ}\cite{Paulson2012ThreeYO}.
  
  The idea of encoding-based translation is to encode
  constructions in the more expressive system using function symbols in the less
  expressive system, and define the translation as a recursive function on the construction
  rules of the more expressive system. For example, in the dependent type theory of Coq,
  we have the type judgement relation $\Gamma \vdash x : w$, which means ``$x$ is of
  type $w$ under context $\Gamma$''. There is no direct equivalent of this
  typing relation in untyped FOL. To express the Coq type judgement in untyped FOL, 
  \cite{Czajka2018HammerFC} first introduces the uninterpreted FOL predicate $T(u^*, a^*)$ for
  first-order terms $u^*$ and $a^*$ translated from Coq term $u$ and atomic Coq type $a$
  (here \textit{atomic} roughly means that $a$ cannot be
  further decomposed by the translation procedure of \cite{Czajka2018HammerFC}). Then, \cite{Czajka2018HammerFC}
  defines the recursive function $\mathcal{G}_\Gamma(u, w)$ on Coq context $\Gamma$ and Coq terms $u, w$.
  The function $\mathcal{G}_\Gamma(u, w)$ translates the typing relation $\Gamma \vdash u : w$ into an untyped FOL formula,
  in which the $T$ predicate is used to express type judgements involving atomic types.
  
  Encoding-based translation has the advantage of being (almost) complete
  and straightforward to compute. However, certain features of the more expressive
  logical system are usually omitted to produce translation results of reasonable size,
  which sacrifices soundness. Moreover, even with this tradeoff, the translation result is usually much larger
  than the original term.

  The idea of monomorphization is that, the proof of many propositions in the more expressive logical
  system can essentially be conducted in the less expressive logical system. In other words,
  many features of the more expressive system are often irrelevant to the proof. For example,
  in polymorphic HOL, given
  \begin{enumerate}
    \item The list map function $\mathsf{List.map} : \forall (\alpha \ \beta : \mathsf{Type}). (\alpha \to \beta) \to \mathsf{List} \ \alpha \to \mathsf{List} \ \beta$
    \item Two lists of natural numbers $xs \ ys : \mathsf{List} \ \mathbb{N}$ and two functions $f \ g : \mathbb{N} \to \mathbb{N}$
    \item The premise $xs = ys \land f = g$
  \end{enumerate}
  The equality
  \begin{equation}\label{lmapphol}
    \mathsf{List.map} \ \mathbb{N} \ \mathbb{N} \ f \ xs = \mathsf{List.append} \ \mathbb{N} \ \mathbb{N} \ g \ ys
  \end{equation}
  is provable using two rewrites $xs \Rightarrow ys, f \Rightarrow g$. The crucial observation is that, although $\textsf{List.map}$ is polymorphic, the term
  $\mathsf{List.map} \ \mathbb{N} \ \mathbb{N}$ as a whole behaves just like a function symbol in monomorphic HOL,
  therefore the rewrite can essentially be performed in monomorphic HOL. More formally,
  the formula \eqref{lmapphol} is the image of the monomorphic HOL formula
  $$h \ f^* \ xs^* = h \ g^* ys^*$$
  under the inter-logical-system "substitution"
  $$\sigma := \{h \mapsto \mathsf{List.map} \ \mathbb{N} \ \mathbb{N},
    f^* \mapsto f, g^* \mapsto g, xs^* \mapsto xs, ys^* \mapsto ys\}$$
  and the rewrites $xs \Rightarrow ys, f \Rightarrow g$ in polymorphic HOL are just manifestations of the
  rewrites $xs^* \Rightarrow ys^*, f^* \Rightarrow g^*$ in monomorphic HOL.
  
  Monomorphization is a sound translation procedure, produces small translation results, and preserves
  term structures during translation. However, monomorphization is incomplete,
  because it has difficulty handling complex term structures such as existential
  type quantifiers and non-leading universal type quantifiers.

  This paper proposes an extension of Sledgehammer's monomorphization procedure
  to dependent type theory. We implemented the translation in Lean4 under the name Lean-auto.
  As mentioned above, completeness is sacrificed for smaller translation
  results. We argue that smaller problem size is crucial to obtaining better performance
  from ATPs, and that our translation can handel real Lean4 use cases despite being incomplete.
\section{Motivating Examples} \label{motex}

In section \ref{exabst} and \ref{exinst}, we will demonstrate the execution of the
quantifier instantiation and $\lambda_\to^*$ abstraction algorithms of our monomorphization procedure on a
concrete example. Note that both the example and the algorithms
here are simplified. Sect. \ref{exqdet} discusses the challenges posed by dependent
type theory and Lean4.

\subsection{$\lambda_\to^*$ Abstraction} \label{exabst}

\begin{figure}
  \begin{CenteredBox}
    \lstinputlisting[style=leanHH]{LeanCode/reverse_map_pretty.inp}
  \end{CenteredBox}
  \caption{Lean4 proof state of a problem involving \textbf{List}} \label{leanlistpretty}
\end{figure}

The Lean4 proof state of the problem we will consider is shown in Figure \ref{leanlistpretty}.
The hypotheses are displayed before $\vdash$, while the goal comes after $\vdash$.
\vmaprev states that \vmap commutes with \vrev, and
\vrevrev states that \vrev is the inverse function of itself.

\begin{figure}
  \begin{CenteredBox}
    \lstinputlisting[style=leanHH]{LeanCode/reverse_map_explicit.inp}
  \end{CenteredBox}
  \caption{Lean4 proof state after quantifier introduction and proof by contradiction, with implicit arguments displayed.
    Note that the equality sign in Figure \ref{leanlistpretty} is a syntactic sugar of
    the polymorphic function \vEq seen here}
  \label{leanlistexplicit}
\end{figure}

Since the problem is already in the $\lambda C$ fragment of Lean, the only
preprocessing step involved here is introducing all the $\forall$ quantifiers
of the goal into the context, and applying proof by contradiction. The resulting proof state
is shown in Figure \ref{leanlistexplicit}.
For clarity, we have displayed the implicit arguments of all the functions.

First, we focus on translating \vneggoal into $\text{HOL}^*$. Following the
discussion of Sect. \ref{subencmon}, we would like to find an $\text{HOL}^*$ formula $\varphi$
and a ``substitution'' $\sigma$ such that the image of $\varphi$ under $\sigma$ is \vneggoal.
We also want the problem to be provable after the translation, so $\varphi$
should preserve as much information in \vneggoal as possible.

Three polymorphic functions: \vEq, \vmap and \vrev, occur in \vneggoal.
Although these functions are polymorphic, instances of these functions
with dependent arguments instantiated could behave like $\text{HOL}^*$ variables
(we will refer to such instances as $\mathit{HOL}^*$ \textit{instances}).
The type constructor \texttt{List} is also not allowed in $\text{HOL}^*$, but
\texttt{List A} and \texttt{List B} behave just like $\text{HOL}^*$ type variables
(we will refer to expressions such as \texttt{List A} and \texttt{List B} as $\mathit{HOL}^*$ \textit{type instances}).
Therefore, we can choose
$$\begin{aligned}
  \varphi := & \ \neg (\mathsf{EqLB} \ (\mathsf{rB} \ (\mathsf{mAB} \ f^* \ (\mathsf{rA} \ \mathit{xs}^*))) \ (\mathsf{mAB} \ f^* \ \mathit{xs}^*)) \\
  \sigma := & \ \{\mathsf{EqLB} \mapsto \texttt{@Eq (List B)}, \ \ \mathsf{mAB} \mapsto \texttt{@map A B}, \\
            & \ \ \mathsf{rA} \mapsto \texttt{@reverse A}, \ \ \mathsf{rB} \mapsto \texttt{@reverse B}, \ \ f^* \mapsto \texttt{f}, \ \ \mathit{xs}^* \mapsto \texttt{xs} \\
            & \ \ \mathsf{LA} \to \texttt{List A}, \ \ \mathsf{LB} \to \texttt{List B}, \ \ \mathsf{A} \to \texttt{A}, \ \ \mathsf{B} \to \texttt{B}\} \\
\end{aligned}$$
where $\mathsf{EqLB} : \mathsf{LB} \to \mathsf{LB} \to \mathsf{Bool}, \
\mathsf{rA} : \mathsf{LA} \to \mathsf{LA}, \ \mathsf{rB} : \mathsf{LB} \to \mathsf{LB}, \ 
\mathsf{mAB} : (\mathsf{A} \to \mathsf{B}) \to \mathsf{LA} \to \mathsf{LB}, \
f^* : \mathsf{A} \to \mathsf{B}, \ \mathit{xs}^* : \mathsf{LA}$.

In a sense, the $\text{HOL}^*$ (type) instances are ``abstracted'' to $\text{HOL}^*$ (type)
variables. Note that the logical rules of $\text{HOL}^*$ are not relevant to this abstraction procedure, only
the term calculus $\lambda_\to^*$ is involved. Therefore, we name this procedure $\lambda_\to^*$ \textit{abstraction}.

However, $\lambda_\to^*$ abstraction is not directly applicable to \vmaprev and
\vrevrev, because dependent arguments of polymorphic functions occurring in them contain
universally quantified variables. Naturally, we would like to instantiate the quantifiers to make $\lambda_\to^*$
abstraction applicable.

\subsection{Quantifier Instantiation} \label{exinst}

To understand how quantifiers should be instantiated, we investigate how they would
be instantiated if we were to prove the goal manually. There are at least two ways we can proceed. We can
either first use \texttt{@map\_reverse A B} to swap the outer \texttt{reverse} with \texttt{map}, then
use \texttt{@reverse\_reverse A} to eliminate \texttt{reverse}; or, first use
\texttt{@map\_reverse A B} to swap the inner \texttt{reverse} with \texttt{map}, then
use \texttt{@reverse\_reverse B} to eliminate \texttt{reverse}. Notice how the dependent
arguments of a function $f$ in the instantiated hypotheses match the dependent
arguments of $f$ in the $\text{HOL}^*$ instances of $f$ in the goal.

Quantifier instantiation in Lean-auto's monomorphization procedure is based
on a matching procedure that reflects the above observation. Given a set of formulas $S$,
the matching procedure first computes the set $M$ of $\text{HOL}^*$ instances occurring
in $S$, then matches type-quantified formulas in $S$ with elements of $M$. For example,
given $S=$\texttt{\{@map\_reverse, @reverse\_reverse, neg\_goal\}}, the set $M$ will be
\texttt{\{@reverse A, @reverse B, @map A B, @Eq (List B)\}}, all of which collected from \vneggoal.
The matching procedure will preform the following matchings:

\begin{enumerate}
  \item \texttt{@Eq (List $\beta$)} in \vmaprev with \texttt{@Eq (List B)},
    which produces \texttt{fun $\alpha$ => @map\_reverse $\alpha$ B}
  \item \texttt{@map $\alpha$ $\beta$} in \vmaprev with \texttt{@map A B},
    which produces \texttt{map\_reverse A B}
  \item \texttt{@reverse $\alpha$} in \vmaprev with \texttt{@reverse A} and \texttt{@reverse B},
    which produces \texttt{@map\_reverse A} and \texttt{@map\_reverse B}
  \item \texttt{@reverse $\beta$} in \vmaprev with \texttt{@reverse A} and \texttt{@reverse B},
    which produces \texttt{fun $\alpha$ => @map\_reverse $\alpha$ A} and \texttt{fun $\alpha$ => @map\_reverse $\alpha$ B}
  \item \texttt{@Eq (List $\alpha$)} in \vrevrev with \texttt{@Eq (List B)},
    which produces \texttt{@reverse\_reverse B}
  \item \texttt{@reverse $\alpha$} in \vrevrev with \texttt{@reverse A} and \texttt{@reverse B},
    which produces \texttt{@reverse\_reverse A} and \texttt{@reverse\_reverse B}
\end{enumerate}

Since \texttt{@reverse\_reverse A}, \texttt{@reverse\_reverse B} and \texttt{@map\_reverse A B}
are present, the instances produced are already sufficient for proving the goal.
But generally speaking, newly generated hypothesis instances and $\text{HOL}^*$ instances
in them can still be matched with each other (and existing ones) to produce new useful results.
% TODO: Give a concrete example taken from my graduation thesis?
Hence, the monomorphization procedure in Lean-auto uses a saturation loop which
repeats the matching procedure until either no new instances can be produced or a prescribed
threshold is reached.

\subsection{Challenges related to Lean4} \label{exqdet}

\noindent \textbf{Dependent Arguments are Dynamic}

\begin{figure}
  \begin{CenteredBox}
    \begin{lstlisting}[style=leanHH]
|!@DFunLike.coe!| : {F : Type (max u_1 u_5)}
  → {α : outParam (Type u_1)} → {β : outParam (α → Type u_5)}
  → [self : DFunLike F α β] → F → (a : α) → β a

@DFunLike.coe (A₀ →+ B₀) A₀ (fun x => B₀) AddMonoidHom.instFunLike f₀ a 
    \end{lstlisting}
  \end{CenteredBox}
  \caption{The function \texttt{DFunLike.coe} from MathLib4 and an expression
  containing it}
  \label{dfun}
\end{figure}

  In $\lambda C$, whether an argument is dependent depends on how previous arguments
are instantiated. Consider the example shown in Figure \ref{dfun}. The return
type \texttt{$\beta$ a} depends on the last argument \texttt{a : $\alpha$} in the signature of
\texttt{DFunLike.coe}. However, when $\beta$ is instantiated with \texttt{fun x => B$_0$}, as in the
expression at the bottom of Figure \ref{dfun}, the return type \texttt{$\beta$ a} reduces to \texttt{B$_0$},
which no longer depends on the last argument. Our monomorphization procedure takes preceding arguments into
consideration when determining whether an argument is dependent.

\noindent \textbf{$\text{HOL}^*$ Instances are Dynamic}

  In $\lambda C$, whether an expression is an $\text{HOL}^*$ instance is context-dependent.
Consider the simple expression \texttt{@reverse = @reverse}, where \texttt{reverse}
is the same as in Figure \ref{leanlistexplicit}. Although \texttt{@reverse} is polymorphic,
it \textit{behaves like} an $\text{HOL}^*$ variable in \texttt{@reverse = @reverse}. More formally,
let
$$\begin{aligned}
\varphi &:= (f = f) \\
\sigma  &:= \{f \mapsto \texttt{@reverse}, \ \ \alpha \mapsto \texttt{(} \forall \ \texttt{\{}\alpha \ \beta \ : \ \texttt{Type\}}, \ 
  \texttt{List} \ \alpha \to \texttt{List} \ \beta \texttt{)}\}
\end{aligned}$$
where $f : \alpha$. Then, \texttt{@reverse = @reverse} is the image of the $\text{HOL}^*$ formula $\varphi$
under $\sigma$. Intuitively, the dependent arguments of \texttt{reverse} can be ``absorbed''
into the $\text{HOL}^*$ type variable $\alpha$ because both dependent arguments of \texttt{reverse} are unapplied.
Our monomorphization procedure is able to detect such context-dependent $\text{HOL}^*$ instances.

\noindent \textbf{Definitional Equality}

  As mentioned before, two syntactically different expressions can be definitionally
equal in Lean4. Theoretically speaking, reducing all expressions to their normal forms will solve
the definitional equality problem to a large extent. However, reduction is prohibitively
expensive on complex expressions in real-life Lean4 projects. Moreover, the reduced expression
could be much larger than the original expression, and might contain complex dependent types
that Lean-auto cannot handle. %TODO: Experiments

  Therefore, we have to find other methods to address definitional equality.
In Lean-auto, there are three separate occasions where definitional equality
has to be addressed.

  First, when a symbol is defined in Lean4, (potentially multiple) \textit{equational theorems} that
reflect the definitional equalities related to the symbol are automatically generated.
Lean-auto can be configured to collect equational theorems, perform reduction and
unfold constants, see Sect. \ref{sectprep}.

  Second, during $\lambda_\to^*$ abstraction, we would like $\text{HOL}^*$ instances
that are syntactically different but definitionally equal to be abstracted to the
same $\text{HOL}^*$ variable. Our $\lambda_\to^*$ abstraction algorithm keeps a
set $H$ of mutually definitionally unequal $\text{HOL}^*$ instances. Whenever a new $\text{HOL}^*$
instance $t$ is found, we test definitional equality of $t$ with elements of $H$
using \texttt{isDefEq}. Since \texttt{isDefEq} is expensive,
a \textit{fingerprint} is computed for each $\text{HOL}^*$ instance, and fingerprint equality
is tested before calling \texttt{isDefEq}.

  Finally, even if two $\text{HOL}^*$ instances are definitionally unequal, there could still be nontrivial
relations between them. For example, suppose $f : \mathbb{N} \to \mathbb{N}$ is defined as
$f := \lambda (x : \mathbb{N}). g \ x \ x$, then $f$ is not definitionally equal to $g$. However, we do
have the nontrivial equation $\forall (x : \mathbb{N}). f \ x = g \ x \ x$. Lean-auto will attempt to generate such equations
after quantifier instantiation completes. For each pair of $\text{HOL}^*$ instances $c_1, c_2$,
Lean-auto tries to find terms $t_1, \dots, t_n$ such that $\forall x_1 \dots x_m. \ c_1 \ y_1 \ \dots \ y_l = c_2 \ t_1 \ \dots \ t_n$,
where $x_1, \dots, x_m$ are variables occurring in $t_1, \dots, t_n$, and
$\{y_1, \dots, y_l\}$ is a subset of $\{x_1, \dots, x_m\}$. For simplicity, this
definitional equality generation procedure is not included in Sect \ref{sectabst} and Sect \ref{sectinst}.

\noindent \textbf{Absorbing Typeclass Instance Arguments}

  In Lean4, many function's instance argument(s) are not dependent arguments,
for example the fourth argument of \texttt{HAdd.hAdd} mentioned in Sect. \ref{sectlean}.
Since instance arguments are usually large expressions synthesized by Lean4's typeclass
inference algorithm, their presence can result in large translation results.
Lean-auto's actual implementation attempts to absorbed typeclass arguments into
$\text{HOL}^*$ variables by instantiating typeclass instance quantifiers
and requiring $\text{HOL}^*$ instances to take typeclass arguments with them. For
simplicity, this detail is not included in Sect \ref{sectabst} and Sect \ref{sectinst}.

\section{$\lambda_\to$ abstraction}

\subsection{$\lambda C$, $\lambda_\to^*$ and $\lambda_\to$}

In this section, we define three type systems: $\lambda C, \lambda_\to^*$ and $\lambda_\to$. $\lambda C$
is calculus of constructions with a countable family of non-cumulative universe levels, similar
to the type theory of Lean; $\lambda_\to^*$ is simply typed lambda calculus with a countable
family of universe levels, without $\mathsf{U}_0$ and without the typing relation
$\mathsf{U}_\ell : \mathsf{U}_{\ell + 1}$, intended to model the output of Lean-auto's
monomorphization; $\lambda_\to$ is simply typed lambda calculus. All three systems will be
specified as pure-type systems $(\mathcal{S}, \mathcal{A}, \mathcal{R})$, where $\mathcal{S}$
is the set of sorts, $\mathcal{A} \subseteq \mathcal{S}^2$ is the set of axioms, and
$\mathcal{R} \subseteq \mathcal{S}^3$ corresponds to the formation rules.

\begin{definition} $\lambda C$ is the pure type system $(\mathcal{S}, \mathcal{A}, \mathcal{R})$ where
  $$\mathcal{S} := \{\mathsf{U}_\ell | \ell \in \mathbb{N}\} \ \ \ \mathcal{A} := \{(\mathsf{U}_\ell, \mathsf{U}_{\ell + 1}) | \ell \in \mathbb{N}\}$$
  $$\mathcal{R} := \{(\mathsf{U}_\ell, \mathsf{U}_m, \mathsf{U}_{\mathsf{imax}(\ell, m)}) | \ell \in \mathbb{N}, m \in \mathbb{N}\}$$
  $$\mathsf{imax}(m, n) := \left\{\begin{aligned}
    \mathsf{max}(m, n), & & n > 0 \\
    0, & & n = 0
  \end{aligned}\right.$$  
\end{definition}

\begin{definition} $\lambda_\to^*$ is the pure type system $(\mathcal{S}, \mathcal{A}, \mathcal{R})$ where
  $$\mathcal{S} := \{\mathsf{U}_\ell | \ell \in \mathbb{N}^*\} \cup \{\mathsf{U}_\ell' | \ell \in \mathbb{N}^*\} \ \ \
    \mathcal{A} := \{(\mathsf{U}_\ell, \mathsf{U}_\ell') | \ell \in \mathbb{N}^*\}$$
  $$\mathcal{R} := \{(\mathsf{U}_\ell, \mathsf{U}_m, \mathsf{U}_{\mathsf{max} \{l, m\}}) | \ell \in \mathbb{N}^*, m \in \mathbb{N}^*\}$$
\end{definition}

\begin{definition} $\lambda_\to$ is the pure type system $(\mathcal{S}, \mathcal{A}, \mathcal{R})$ where
  $$\mathcal{S} := \{\mathsf{U}_1, \mathsf{U}_1'\} \ \ \ \mathcal{A} := \{(\mathsf{U}_1, \mathsf{U}_1')\} \ \ \ 
    \mathcal{R} := \{\mathsf{U}_1, \mathsf{U}_1, \mathsf{U}_1\}$$
  This is equivalent to simply typed lambda calculus, where $\mathsf{U}_1$ and $\mathsf{U}_1'$ are
  usually denoted as $*$ and $\square$, respectively.
\end{definition}

\begin{theorem} A $\lambda_\to^*$ problem is provable iff it is $\lambda_\to$ provable after forgetting
  universe levels, assuming the existence of lifting functions in $\lambda C$.
\end{theorem}
\begin{proof} \textbf{TODO}
  (Put this after the definition of provability)
  (Maybe a direct induction would be more elegant?)
  If a $\lambda_\to^*$ problem is provable, obviously it is $\lambda_\to$ provable
  after forgetting the universe levels. If a $\lambda_\to^*$ problem is $\lambda_\to$ provable
  after forgetting the universe levels, note that the lifting of a $\lambda_\to^*$ problem is sort of
  an instance of the problem after forgetting universe levels, hence provable. Moreover,
  a $\lambda_\to^*$ problem is provable iff its lifting is provable.  
\end{proof}

\subsection{Essentially higher-order problem}

\begin{definition} Let $\sigma : V \to \mathcal{T}$ be a mapping.
  Define its extension $\overline{\sigma} : \mathcal{T} \to \mathcal{T}$ as
  $$\overline{\sigma}(\mathsf{U}_\ell) := \mathsf{U}_\ell$$
  $$\overline{\sigma}(x) := \sigma(x), \text{ for }x \in V$$
  $$\overline{\sigma}(M \ N) := \overline{\sigma}(M) \ \overline{\sigma}(M)$$
  $$\overline{\sigma}(\lambda x : s. M) := \lambda x : \overline{\sigma}(s). \overline{\sigma[x \mapsto x]}(M)$$
  $$\overline{\sigma}(\forall x : s. M) := \forall x : \overline{\sigma}(s). \overline{\sigma[x \mapsto x]}(M)$$
  where
  $$\sigma[u \to t](x) := \left\{\begin{aligned}
    & t & & x = u \\
    & \sigma(x) & & x \in V \backslash \{u\}
  \end{aligned}\right.$$
\end{definition}

\begin{definition} A substitution is a triple $(\Gamma, \Gamma', \sigma)$ where $\Gamma, \Gamma'$ are contexts
  and $\sigma : V \to \mathcal{T}$, such that for all $(u : \tau) \in \Gamma$,
  $$\Gamma' \vdash \sigma(u) : \overline{\sigma}(\tau)$$
  $\Gamma$ is called the domain of the substitution, and $\Gamma'$ is called the codomain of the substitution.
\end{definition}

\begin{theorem}
  Let $(\Gamma, \Gamma', \sigma)$ be a substitution. If $\Gamma \vdash t : s$, then $\Gamma' \vdash \overline{\sigma}(t) : \overline{\sigma}(s)$
\end{theorem}
\begin{proof} Induction on the derivation of $\Gamma \vdash t : s$, note that $\Gamma'$ and $\sigma$
  should be universally quantified in the induction hypothesis.
\end{proof}

\begin{definition} Many-sorted higher-order logic (HOL) is defined as $\lambda_\to^*$ augmented
  with the following symbols:
  \begin{enumerate}
    \item $\mathsf{Bool}$
    \item $\bot'$ and $\to'$
    \item $\forall'_t$, for each $t \in \mathcal{T}$
  \end{enumerate}
  
  \noindent and the following derivation rules:
  $$\frac{}{\vdash \mathsf{Bool} : \mathsf{U}_1} \ \ \ \ \frac{}{\Gamma \vdash \bot' : \mathsf{Bool}}$$
  $$\frac{}{\Gamma \vdash \to' : \mathsf{Bool} \to \mathsf{Bool} \to \mathsf{Bool}} \ \ \ \
  \frac{\Gamma \vdash s : \mathsf{U}_\ell}{\Gamma \vdash \forall'_s : (s \to \mathsf{Bool}) \to \mathsf{Bool}}$$
  
  \noindent For simplicity, we use $\forall' (x : \alpha), t$ as a shorthand for $\forall'_\alpha \ (\lambda x : \alpha. t)$

  \noindent The canonical embedding $\pi : \mathcal{T}_{\text{HOL}} \to \mathcal{T}$ of HOL into $\lambda C$ is defined as follows:
  $$\pi(\mathsf{Bool}) := \mathsf{U}_0 \ \ \ \pi(\mathsf{U}_\ell) := \mathsf{U}_\ell \ \ \
    \pi(\mathsf{U}_\ell') := \mathsf{U}_{\ell + 1}$$
  $$\pi(\bot') := \forall (\alpha : \mathsf{U}_0). \alpha \ \ \
  \pi(\to') := \lambda (p \ q : \mathsf{U}_0). \forall (x : p). q$$
  $$\pi(\forall_t') := \lambda (p : \pi(t) \to \mathsf{U}_0). \forall (x : \pi(t)). p \ x$$
  $$\pi(x) := x, \text{ for } x \in V$$
  $$\pi(M \ N) := \pi(M) \ \pi(N) \ \ \ \pi(\lambda x : s. M) := \lambda x : \pi(s). \pi(M)$$

  \noindent $\pi$ is extended to contexts via the following definition:
  $$\pi(\emptyset) := \emptyset \ \ \ \pi(\Gamma, x : \sigma) := \pi(\Gamma), x : \pi(\sigma)$$

  \noindent \textbf{TODO: Is this really equivalent to higher-order logic?}

\end{definition}

\begin{theorem}\label{ceptj} Canonical embedding preserves judgement, i.e. if $\Gamma \vdash t : s$ in HOL, then
  $\pi(\Gamma) \vdash \pi(t) : \pi(s)$ in $\lambda C$ \end{theorem}
\begin{proof} Induction on the derivation rules of HOL. \end{proof}

\begin{definition} An (HOL/$\lambda C$) problem is a tuple $(\Gamma, p)$, denoted
  as $\Gamma \vdash? p$, where $\Gamma$ is a (HOL/$\lambda C$)
  context, called the hypotheses of the problem, and $p$ is an
  (HOL/$\lambda C$) term, called the goal of the problem. A $\lambda C$ problem
  $\Gamma \vdash? p$ is provable iff there exists a $\lambda C$ term $t$ such that
  $\Gamma \vdash t : p$. An HOL problem $\Gamma \vdash? p$ is provable iff there exists
  a $\lambda C$ term $t$ such that $\pi(\Gamma) \vdash t : \pi(p)$.
\end{definition}

\begin{definition} A $\lambda C$ problem $\Gamma \vdash? p$ is essentially higher-order provable (EHOP)
  iff there exists a provable HOL problem $\Gamma' \vdash? p'$ and a substitution
  $(\pi(\Gamma'), \Gamma, \sigma)$ such that $p = \overline{\sigma}(\pi(p'))$.
\end{definition}

\begin{theorem}
  If a $\lambda C$ problem $\Gamma \vdash? p$ is EHOP, then it is provable.
\end{theorem}
\begin{proof} By the definition of EHOP, there exists a provable HOL problem
  $\Gamma' \vdash? p'$ and substitution $(\pi(\Gamma'), \Gamma, \sigma)$ such that
  $p = \overline{\sigma}(\pi(p'))$. By the definition of HOL provability, there exists
  a term $t'$ such that $\pi(\Gamma') \vdash t' : \pi(p')$. By theorem \ref{ceptj},
  $\Gamma \vdash \overline{\sigma}(t') : \overline{\sigma}(\pi(p'))$, thus $\Gamma \vdash? p$
  is provable.
\end{proof}

\noindent We define logical symbols in $\lambda C$ as follows:
$$\bot := \forall p : \mathsf{U}_0. p \ \ \ (\neg) := \lambda p : \mathsf{U}_0. p \to \bot$$
$$(\land) := \lambda p \ q : \mathsf{U}_0. \forall r : \mathsf{U}_0. (p \to q \to r) \to r$$
$$(\lor) := \lambda p \ q : \mathsf{U}_0. \forall r : \mathsf{U}_0. (p \to r) \to (q \to r) \to r$$
$$(\leftrightarrow) := \lambda p \ q. (p \to q) \land (q \to p)$$
$$(=_\ell) := \lambda \alpha : \mathsf{U}_\ell. \lambda x \ y : \alpha. \forall p : \alpha \to \mathsf{U}_0. (p \ x \leftrightarrow p \ y)$$
$$(\exists_\ell) := \lambda \alpha : \mathsf{U}_\ell. \lambda p : \alpha \to \mathsf{U}_0. \forall q : \mathsf{U}_0. ((\forall x : \alpha. p \ x \to q) \to q)$$

\noindent The symbols $\neg', \land', \lor', \leftrightarrow'$ are defined in HOL in the
same way, except that the $\mathsf{U}_0$s are replaced with $\mathsf{Bool}$ and the $\to$s are
replaced with $\to'$. Equality and existential quantifier in HOL are defined as follows:
$$(=_s') := \lambda x \ y : s. \forall' p : \alpha \to \mathsf{Bool}. (p \ x \leftrightarrow' p \ y)$$
$$(\exists_s') := \lambda p : \alpha \to \mathsf{Bool}. \forall' q : \mathsf{Bool}. ((\forall' x : \alpha. p \ x \to' q) \to' q)$$

\noindent We also assume that excluded middle, i.e. $\mathsf{em} : \forall p : \mathsf{U}_0, p \lor \neg p$,
is implicitly contained in the hypotheses of all $\lambda C$ problems. Similarly, $\mathsf{em}' : \forall p : \mathsf{Bool}, p \lor \neg p$
is assumed to be implicitly contained in the hypotheses of all HOL problems.

\begin{example} Consider the $\lambda C$ problem $\Gamma \vdash? p$ where
\begin{align*}
  \Gamma := \ & \mathbb{N} : \mathsf{U}_1, \mathsf{Fin} : \mathbb{N} \to \mathsf{U}_1,
  \mathsf{add} : \forall n : \mathbb{N}, (\mathsf{Fin} \ n \to \mathsf{Fin} \ n \to \mathsf{Fin} \ n), n : \mathbb{N} \\
  p := \ & (\forall (u \ v : \mathsf{Fin} \ n). \mathsf{add} \ n \ u \ v = \mathsf{add} \ n \ v \ u) \to \\
  & \ \ \ \forall (u \ v \ w : \mathsf{Fin} \ n). \mathsf{add} \ n \ (\mathsf{add} \ n \ x \ y) \ z = \mathsf{add} \ n \ z \ (\mathsf{add} \ n \ y \ x)
\end{align*}
Given
\begin{align*}
  \Gamma' := \ & \alpha : \mathsf{U}_1, f : \alpha \to \alpha \to \alpha \\
  p' := \ & (\forall' (u \ v : \alpha). f \ u \ v =_\alpha' f \ v \ u) \to' \\
  & \ \ \ \forall' (u \ v \ w : \alpha), f \ (f \ u \ v) \ w =_\alpha' f \ w \ (f \ v \ u)
\end{align*}
The higher-order problem $\Gamma' \vdash? p'$ is provable. Moreover, given
$$\sigma(\alpha) := \mathsf{Fin} \ n, \sigma(f) := \mathsf{add} \ n$$
The triple $(\pi(\Gamma'), \Gamma, \sigma)$ forms a substitution, and $p = \overline{\sigma}(\pi(p'))$.
Therefore, $\Gamma \vdash? p$ is EHOP.
\end{example}

\noindent Note that moving implications in the goal into hypotheses (and vice versa) may
change the EHOP status of a problem. For example,
$$\alpha : \mathsf{U}_1, x : \alpha, p : \alpha \to \mathsf{U}_0 \vdash? p \ x \to p \ x$$
is EHOP. However, if we introduce $p \ x$ into the hypotheses, the problem is no longer EHOP:
\begin{equation}\label{hypnehop}
  \alpha : \mathsf{U}_1, x : \alpha, p : \alpha \to \mathsf{U}_0, h : p \ x \vdash? p \ x
\end{equation}

\begin{theorem}
  The $\lambda C$ problem \eqref{hypnehop} is provable but not EHOP.
\end{theorem}
\begin{proof}
  Note that $h : p \ x$ under the hypotheses of \eqref{hypnehop}, thus \eqref{hypnehop} is provable.
  To show that \eqref{hypnehop} is not EHOP, we use proof by contradiction. Suppose there
  is an HOL problem $\Gamma' \vdash p'$ and a substitution $(\Gamma', \Gamma, \sigma)$ such
  that $p = \overline{\sigma}(\pi(p'))$. Then, the $\beta$-nf of $p'$ must be of the form
  $f \ t_1 \ \dots \ t_k$ where $f$ is a variable. Note that $\Gamma'$, as a context of $\lambda_\to^*$,
  consists solely of HOL (type or term) variable declarations. \textbf{TODO:} Use model theory
\end{proof}

\section{Quantifier Instantiation}
Given a context $\Gamma$ and a list of hypotheses $h_1 : t_1, \dots, h_n : t_n$, the quantifier
instantiation procedure of Lean-auto attempts to instantiate quantifiers in
$t_1, \dots, t_n$ to obtain terms suitable for $\lambda_\to^*$ abstraction.
As mentioned in section \ref{motex}, the quantifier instantiation procedure
is based on a saturation loop which matches instances of polymorphic functions
with subterms of hypothesis instances.

\noindent An instance of a polymorphic function is called \textit{monomorphic} iff all
of its dependent arguments are instantiated with terms that do not contain bound
variables. Formally, the set of monomorphic instances of polymorphic functions in a
$\lambda C$ term is defined as follows:

\begin{definition}
  Let $\Gamma$ be a $\lambda C$ context and $V$ be a set of variables, then
  \begin{enumerate}
    \item For variable $x$ and terms $t_1, \dots, t_n$,
      $$\mathsf{monoInst}(\Gamma; V, x \ t_1 \ \dots \ t_n) := \left\{
        \begin{aligned}
          S \cup \{l\}, & & FV(l) \cap V = \emptyset \\
          S, & & \text{otherwise}
        \end{aligned}
      \right.$$
      where
      $$l := \mathsf{LFun}(\Gamma; x, (t_1 \ \dots \ t_n)) \ \ \ \ \ \ S := \bigcup_{t \in \mathsf{LArgs}(\Gamma; x, (t_1, \dots, t_n))} \mathsf{monoInst}(\Gamma; V, t)$$
    \item For variable $x$ and terms $a, b$,
      \begin{align*}
        \mathsf{monoInst}(\Gamma; V, \forall (x : a). b) = \mathsf{monoInst}(\Gamma; V, \lambda (x : a). b) 
        \\ := \mathsf{monoInst}(\Gamma; V, a) \cup \mathsf{monoInst}(\Gamma, x : a; V \cup \{x\}, b)
      \end{align*}
    \item Otherwise, $\mathsf{monoInst}(\Gamma; V, t) := \emptyset$
  \end{enumerate}
\end{definition}

\noindent The matching procedure in the saturation loop is handled by $\mathsf{matchInst}$ and
$\mathsf{match}$. Given context $\Gamma$, variable set $M$ and terms $m, h$
\begin{enumerate}
  \item $\mathsf{match}(\Gamma; M, m, h)$ returns all $M$-unifiers between term $m$ and the $\mathsf{LFun}$ of subterms of $h$.
    The pseudocode for $\mathsf{match}$ is given in \textbf{Algorithm \ref{matching}}. An auxiliary function
    $\mathsf{unify}$ is used in the pseudocode. Given $\lambda C$ context $\Gamma$, variable set $M$
    and two $\lambda C$ terms $t_1, t_2$, $\mathsf{union}(\Gamma; M, t_1, t_2)$ returns a complete set of
    $M$-unifiers of $t_1$ and $t_2$ under $\Gamma$. In Lean, the function $\mathsf{Meta.isDefEq}$ is the closest
    to the $\mathsf{union}$ here, but $\mathsf{Meta.isDefEq}$ always returns at most one unifier.
  \item $\mathsf{matchInst}(\Gamma; m, h)$ introduces all leading non-prop $\forall$ quantifiers into the context
    (as free variables), collects all the newly introduced free variables into a variable set $M$, then returns
    $\mathsf{match}(\Gamma'; M, m, h')$, where $\Gamma', h'$ are $\Gamma, h$ after introduction of free variables.
    In Lean, this is equivalent to calling $\mathsf{match}$ after instantiating all the leading non-prop
    $\forall$ quantifiers of $h$ with metavariables.
\end{enumerate}

\textbf{TODO:} Non-subsumption equivalence; Saturation loop

\begin{algorithm}\label{matching}
  \DontPrintSemicolon
  \SetNoFillComment
  \SetKwFunction{matchFun}{\textsf{match}}
  \caption{Matching algorithm for quantifier instantiation}
  \Fn{\matchFun{$\Gamma; M, m, h$}}{
    \Input{$\lambda C$ context $\Gamma$, variable set $M$, and $\lambda C$ terms $m, h$}
    \Output{A set of unifiers}
    \Switch(\textbf{with}){h}{
      \Case(\tcc*[h]{Function application}){$a \ b$}{
        $matches := \emptyset$ \;
        $f := \mathsf{getAppFn}(t)$ \;
        $args := \mathsf{getAppArgs}(t)$ \;
        \For{a : args}{$matches := \mathsf{union}(matches, \mathsf{match}(\Gamma; M, m, a))$}
        $lf := \mathsf{LFun}(\Gamma; f, arg)$ \;
        $matches := \mathsf{union}(matches, \mathsf{unify}(\Gamma; M, m, lf))$
      }
      \Case{$\forall (v : a). b$}{
        \Return $\mathsf{union}(\mathsf{match}(\Gamma; M, m, a), \mathsf{match}(\Gamma, v : a; M, m, b))$
      }
      \Case{$\lambda (v : a). b$}{
        \Return $\mathsf{union}(\mathsf{match}(\Gamma; M, m, a), \mathsf{match}(\Gamma, v : a; M, m, b))$
      }
      \Other{\Return $\emptyset$}
    }
  }
\end{algorithm}
% \input{example}
% \input{credits}
\input{bibliography}

\end{document}
