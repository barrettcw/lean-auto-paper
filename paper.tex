% This is samplepaper.tex, a sample chapter demonstrating the
% LLNCS macro package for Springer Computer Science proceedings;
% Version 2.21 of 2022/01/12
%
\documentclass[runningheads]{llncs}
%
\usepackage[T1]{fontenc}
% T1 fonts will be used to generate the final print and online PDFs,
% so please use T1 fonts in your manuscript whenever possible.
% Other font encondings may result in incorrect characters.
%
\usepackage{graphicx}
% Used for displaying a sample figure. If possible, figure files should
% be included in EPS format.
%
% If you use the hyperref package, please uncomment the following two lines
% to display URLs in blue roman font according to Springer's eBook style:
%\usepackage{color}
%\renewcommand\UrlFont{\color{blue}\rmfamily}
%\urlstyle{rm}
%
% Packages required for listing Lean code
\usepackage[T1]{fontenc}
\usepackage[utf8]{inputenc}
\usepackage{listings}
\usepackage{color}
\def\lstlanguagefiles{lstlean.tex}
\definecolor{hypcolor}{rgb}{0.85, 0.48, 0.0}
\lstdefinestyle{leanHH}{
  language=lean,
  moredelim=**[is][\color{hypcolor}]{|!}{!|},
}
\usepackage{amssymb}
\newsavebox\vEq
\newsavebox\vmap
\newsavebox\vrev
\newsavebox\vmaprev
\newsavebox\vrevrev
%\definecolor{keywordcolor}{rgb}{0.7, 0.1, 0.1}   % red
\definecolor{keywordcolor}{rgb}{0.0, 0.0, 0.0}   % black
\definecolor{tacticcolor}{rgb}{0.0, 0.3, 0.8}    % blue
\definecolor{commentcolor}{rgb}{0.4, 0.4, 0.4}   % grey
%\definecolor{symbolcolor}{rgb}{0.0, 0.1, 0.6}    % blue
\definecolor{symbolcolor}{rgb}{0.0, 0.0, 0.0}    % black
%\definecolor{sortcolor}{rgb}{0.1, 0.5, 0.1}      % green
\definecolor{sortcolor}{rgb}{0.0, 0.0, 0.0}      % black
\definecolor{attributecolor}{rgb}{0.7, 0.1, 0.1} % red

\usepackage{fancybox}
\makeatletter
\newenvironment{CenteredBox}{% 
\begin{Sbox}}{% Save the content in a box
\end{Sbox}\centerline{\parbox{\wd\@Sbox}{\TheSbox}}}% And output it centered
\makeatother

\usepackage{amsmath}


\begin{document}
%
\title{Lean auto: An interface between Lean and Automated Theorem Provers}
%
%\titlerunning{Abbreviated paper title}
% If the paper title is too long for the running head, you can set
% an abbreviated paper title here
%
\author{First Author\inst{1}\orcidID{0000-1111-2222-3333} \and
Second Author\inst{2,3}\orcidID{1111-2222-3333-4444} \and
Third Author\inst{3}\orcidID{2222--3333-4444-5555}}
%
\authorrunning{F. Author et al.}
% First names are abbreviated in the running head.
% If there are more than two authors, 'et al.' is used.
%
\institute{Princeton University, Princeton NJ 08544, USA \and
Springer Heidelberg, Tiergartenstr. 17, 69121 Heidelberg, Germany
\email{lncs@springer.com}\\
\url{http://www.springer.com/gp/computer-science/lncs} \and
ABC Institute, Rupert-Karls-University Heidelberg, Heidelberg, Germany\\
\email{\{abc,lncs\}@uni-heidelberg.de}}
%
\maketitle              % typeset the header of the contribution
%
\begin{abstract}
The abstract should briefly summarize the contents of the paper in
150--250 words.

\keywords{First keyword  \and Second keyword \and Another keyword.}
\end{abstract}
%
%
%

\section{Motivating Examples} \label{motex}

In section \ref{exabst} and \ref{exinst}, we will demonstrate the execution of the
quantifier instantiation and $\lambda_\to^*$ abstraction algorithms of our monomorphization procedure on a
concrete example. Note that both the example and the algorithms
here are simplified. Sect. \ref{exqdet} discusses the challenges posed by dependent
type theory and Lean4.

\subsection{$\lambda_\to^*$ Abstraction} \label{exabst}

\begin{figure}
  \begin{CenteredBox}
    \lstinputlisting[style=leanHH]{LeanCode/reverse_map_pretty.inp}
  \end{CenteredBox}
  \caption{Lean4 proof state of a problem involving \textbf{List}} \label{leanlistpretty}
\end{figure}

The Lean4 proof state of the problem we will consider is shown in Figure \ref{leanlistpretty}.
The hypotheses are displayed before $\vdash$, while the goal comes after $\vdash$.
\vmaprev states that \vmap commutes with \vrev, and
\vrevrev states that \vrev is the inverse function of itself.

\begin{figure}
  \begin{CenteredBox}
    \lstinputlisting[style=leanHH]{LeanCode/reverse_map_explicit.inp}
  \end{CenteredBox}
  \caption{Lean4 proof state after quantifier introduction and proof by contradiction, with implicit arguments displayed.
    Note that the equality sign in Figure \ref{leanlistpretty} is a syntactic sugar of
    the polymorphic function \vEq seen here}
  \label{leanlistexplicit}
\end{figure}

Since the problem is already in the $\lambda C$ fragment of Lean, the only
preprocessing step involved here is introducing all the $\forall$ quantifiers
of the goal into the context, and applying proof by contradiction. The resulting proof state
is shown in Figure \ref{leanlistexplicit}.
For clarity, we have displayed the implicit arguments of all the functions.

First, we focus on translating \vneggoal into $\text{HOL}^*$. Following the
discussion of Sect. \ref{subencmon}, we would like to find an $\text{HOL}^*$ formula $\varphi$
and a ``substitution'' $\sigma$ such that the image of $\varphi$ under $\sigma$ is \vneggoal.
We also want the problem to be provable after the translation, so $\varphi$
should preserve as much information in \vneggoal as possible.

Three polymorphic functions: \vEq, \vmap and \vrev, occur in \vneggoal.
Although these functions are polymorphic, instances of these functions
with dependent arguments instantiated could behave like $\text{HOL}^*$ variables
(we will refer to such instances as $\mathit{HOL}^*$ \textit{instances}).
The type constructor \texttt{List} is also not allowed in $\text{HOL}^*$, but
\texttt{List A} and \texttt{List B} behave just like $\text{HOL}^*$ type variables
(we will refer to expressions such as \texttt{List A} and \texttt{List B} as $\mathit{HOL}^*$ \textit{type instances}).
Therefore, we can choose
$$\begin{aligned}
  \varphi := & \ \neg (\mathsf{EqLB} \ (\mathsf{rB} \ (\mathsf{mAB} \ f^* \ (\mathsf{rA} \ \mathit{xs}^*))) \ (\mathsf{mAB} \ f^* \ \mathit{xs}^*)) \\
  \sigma := & \ \{\mathsf{EqLB} \mapsto \texttt{@Eq (List B)}, \ \ \mathsf{mAB} \mapsto \texttt{@map A B}, \\
            & \ \ \mathsf{rA} \mapsto \texttt{@reverse A}, \ \ \mathsf{rB} \mapsto \texttt{@reverse B}, \ \ f^* \mapsto \texttt{f}, \ \ \mathit{xs}^* \mapsto \texttt{xs} \\
            & \ \ \mathsf{LA} \to \texttt{List A}, \ \ \mathsf{LB} \to \texttt{List B}, \ \ \mathsf{A} \to \texttt{A}, \ \ \mathsf{B} \to \texttt{B}\} \\
\end{aligned}$$
where $\mathsf{EqLB} : \mathsf{LB} \to \mathsf{LB} \to \mathsf{Bool}, \
\mathsf{rA} : \mathsf{LA} \to \mathsf{LA}, \ \mathsf{rB} : \mathsf{LB} \to \mathsf{LB}, \ 
\mathsf{mAB} : (\mathsf{A} \to \mathsf{B}) \to \mathsf{LA} \to \mathsf{LB}, \
f^* : \mathsf{A} \to \mathsf{B}, \ \mathit{xs}^* : \mathsf{LA}$.

In a sense, the $\text{HOL}^*$ (type) instances are ``abstracted'' to $\text{HOL}^*$ (type)
variables. Note that the logical rules of $\text{HOL}^*$ are not relevant to this abstraction procedure, only
the term calculus $\lambda_\to^*$ is involved. Therefore, we name this procedure $\lambda_\to^*$ \textit{abstraction}.

However, $\lambda_\to^*$ abstraction is not directly applicable to \vmaprev and
\vrevrev, because dependent arguments of polymorphic functions occurring in them contain
universally quantified variables. Naturally, we would like to instantiate the quantifiers to make $\lambda_\to^*$
abstraction applicable.

\subsection{Quantifier Instantiation} \label{exinst}

To understand how quantifiers should be instantiated, we investigate how they would
be instantiated if we were to prove the goal manually. There are at least two ways we can proceed. We can
either first use \texttt{@map\_reverse A B} to swap the outer \texttt{reverse} with \texttt{map}, then
use \texttt{@reverse\_reverse A} to eliminate \texttt{reverse}; or, first use
\texttt{@map\_reverse A B} to swap the inner \texttt{reverse} with \texttt{map}, then
use \texttt{@reverse\_reverse B} to eliminate \texttt{reverse}. Notice how the dependent
arguments of a function $f$ in the instantiated hypotheses match the dependent
arguments of $f$ in the $\text{HOL}^*$ instances of $f$ in the goal.

Quantifier instantiation in Lean-auto's monomorphization procedure is based
on a matching procedure that reflects the above observation. Given a set of formulas $S$,
the matching procedure first computes the set $M$ of $\text{HOL}^*$ instances occurring
in $S$, then matches type-quantified formulas in $S$ with elements of $M$. For example,
given $S=$\texttt{\{@map\_reverse, @reverse\_reverse, neg\_goal\}}, the set $M$ will be
\texttt{\{@reverse A, @reverse B, @map A B, @Eq (List B)\}}, all of which collected from \vneggoal.
The matching procedure will preform the following matchings:

\begin{enumerate}
  \item \texttt{@Eq (List $\beta$)} in \vmaprev with \texttt{@Eq (List B)},
    which produces \texttt{fun $\alpha$ => @map\_reverse $\alpha$ B}
  \item \texttt{@map $\alpha$ $\beta$} in \vmaprev with \texttt{@map A B},
    which produces \texttt{map\_reverse A B}
  \item \texttt{@reverse $\alpha$} in \vmaprev with \texttt{@reverse A} and \texttt{@reverse B},
    which produces \texttt{@map\_reverse A} and \texttt{@map\_reverse B}
  \item \texttt{@reverse $\beta$} in \vmaprev with \texttt{@reverse A} and \texttt{@reverse B},
    which produces \texttt{fun $\alpha$ => @map\_reverse $\alpha$ A} and \texttt{fun $\alpha$ => @map\_reverse $\alpha$ B}
  \item \texttt{@Eq (List $\alpha$)} in \vrevrev with \texttt{@Eq (List B)},
    which produces \texttt{@reverse\_reverse B}
  \item \texttt{@reverse $\alpha$} in \vrevrev with \texttt{@reverse A} and \texttt{@reverse B},
    which produces \texttt{@reverse\_reverse A} and \texttt{@reverse\_reverse B}
\end{enumerate}

Since \texttt{@reverse\_reverse A}, \texttt{@reverse\_reverse B} and \texttt{@map\_reverse A B}
are present, the instances produced are already sufficient for proving the goal.
But generally speaking, newly generated hypothesis instances and $\text{HOL}^*$ instances
in them can still be matched with each other (and existing ones) to produce new useful results.
% TODO: Give a concrete example taken from my graduation thesis?
Hence, the monomorphization procedure in Lean-auto uses a saturation loop which
repeats the matching procedure until either no new instances can be produced or a prescribed
threshold is reached.

\subsection{Challenges related to Lean4} \label{exqdet}

\noindent \textbf{Dependent Arguments are Dynamic}

\begin{figure}
  \begin{CenteredBox}
    \begin{lstlisting}[style=leanHH]
|!@DFunLike.coe!| : {F : Type (max u_1 u_5)}
  → {α : outParam (Type u_1)} → {β : outParam (α → Type u_5)}
  → [self : DFunLike F α β] → F → (a : α) → β a

@DFunLike.coe (A₀ →+ B₀) A₀ (fun x => B₀) AddMonoidHom.instFunLike f₀ a 
    \end{lstlisting}
  \end{CenteredBox}
  \caption{The function \texttt{DFunLike.coe} from MathLib4 and an expression
  containing it}
  \label{dfun}
\end{figure}

  In $\lambda C$, whether an argument is dependent depends on how previous arguments
are instantiated. Consider the example shown in Figure \ref{dfun}. The return
type \texttt{$\beta$ a} depends on the last argument \texttt{a : $\alpha$} in the signature of
\texttt{DFunLike.coe}. However, when $\beta$ is instantiated with \texttt{fun x => B$_0$}, as in the
expression at the bottom of Figure \ref{dfun}, the return type \texttt{$\beta$ a} reduces to \texttt{B$_0$},
which no longer depends on the last argument. Our monomorphization procedure takes preceding arguments into
consideration when determining whether an argument is dependent.

\noindent \textbf{$\text{HOL}^*$ Instances are Dynamic}

  In $\lambda C$, whether an expression is an $\text{HOL}^*$ instance is context-dependent.
Consider the simple expression \texttt{@reverse = @reverse}, where \texttt{reverse}
is the same as in Figure \ref{leanlistexplicit}. Although \texttt{@reverse} is polymorphic,
it \textit{behaves like} an $\text{HOL}^*$ variable in \texttt{@reverse = @reverse}. More formally,
let
$$\begin{aligned}
\varphi &:= (f = f) \\
\sigma  &:= \{f \mapsto \texttt{@reverse}, \ \ \alpha \mapsto \texttt{(} \forall \ \texttt{\{}\alpha \ \beta \ : \ \texttt{Type\}}, \ 
  \texttt{List} \ \alpha \to \texttt{List} \ \beta \texttt{)}\}
\end{aligned}$$
where $f : \alpha$. Then, \texttt{@reverse = @reverse} is the image of the $\text{HOL}^*$ formula $\varphi$
under $\sigma$. Intuitively, the dependent arguments of \texttt{reverse} can be ``absorbed''
into the $\text{HOL}^*$ type variable $\alpha$ because both dependent arguments of \texttt{reverse} are unapplied.
Our monomorphization procedure is able to detect such context-dependent $\text{HOL}^*$ instances.

\noindent \textbf{Definitional Equality}

  As mentioned before, two syntactically different expressions can be definitionally
equal in Lean4. Theoretically speaking, reducing all expressions to their normal forms will solve
the definitional equality problem to a large extent. However, reduction is prohibitively
expensive on complex expressions in real-life Lean4 projects. Moreover, the reduced expression
could be much larger than the original expression, and might contain complex dependent types
that Lean-auto cannot handle. %TODO: Experiments

  Therefore, we have to find other methods to address definitional equality.
In Lean-auto, there are three separate occasions where definitional equality
has to be addressed.

  First, when a symbol is defined in Lean4, (potentially multiple) \textit{equational theorems} that
reflect the definitional equalities related to the symbol are automatically generated.
Lean-auto can be configured to collect equational theorems, perform reduction and
unfold constants, see Sect. \ref{sectprep}.

  Second, during $\lambda_\to^*$ abstraction, we would like $\text{HOL}^*$ instances
that are syntactically different but definitionally equal to be abstracted to the
same $\text{HOL}^*$ variable. Our $\lambda_\to^*$ abstraction algorithm keeps a
set $H$ of mutually definitionally unequal $\text{HOL}^*$ instances. Whenever a new $\text{HOL}^*$
instance $t$ is found, we test definitional equality of $t$ with elements of $H$
using \texttt{isDefEq}. Since \texttt{isDefEq} is expensive,
a \textit{fingerprint} is computed for each $\text{HOL}^*$ instance, and fingerprint equality
is tested before calling \texttt{isDefEq}.

  Finally, even if two $\text{HOL}^*$ instances are definitionally unequal, there could still be nontrivial
relations between them. For example, suppose $f : \mathbb{N} \to \mathbb{N}$ is defined as
$f := \lambda (x : \mathbb{N}). g \ x \ x$, then $f$ is not definitionally equal to $g$. However, we do
have the nontrivial equation $\forall (x : \mathbb{N}). f \ x = g \ x \ x$. Lean-auto will attempt to generate such equations
after quantifier instantiation completes. For each pair of $\text{HOL}^*$ instances $c_1, c_2$,
Lean-auto tries to find terms $t_1, \dots, t_n$ such that $\forall x_1 \dots x_m. \ c_1 \ y_1 \ \dots \ y_l = c_2 \ t_1 \ \dots \ t_n$,
where $x_1, \dots, x_m$ are variables occurring in $t_1, \dots, t_n$, and
$\{y_1, \dots, y_l\}$ is a subset of $\{x_1, \dots, x_m\}$. For simplicity, this
definitional equality generation procedure is not included in Sect \ref{sectabst} and Sect \ref{sectinst}.

\noindent \textbf{Absorbing Typeclass Instance Arguments}

  In Lean4, many function's instance argument(s) are not dependent arguments,
for example the fourth argument of \texttt{HAdd.hAdd} mentioned in Sect. \ref{sectlean}.
Since instance arguments are usually large expressions synthesized by Lean4's typeclass
inference algorithm, their presence can result in large translation results.
Lean-auto's actual implementation attempts to absorbed typeclass arguments into
$\text{HOL}^*$ variables by instantiating typeclass instance quantifiers
and requiring $\text{HOL}^*$ instances to take typeclass arguments with them. For
simplicity, this detail is not included in Sect \ref{sectabst} and Sect \ref{sectinst}.

\section{$\lambda_\to$ abstraction}

\subsection{$\lambda C$, $\lambda_\to^*$ and $\lambda_\to$}

In this section, we define three type systems: $\lambda C, \lambda_\to^*$ and $\lambda_\to$. $\lambda C$
is calculus of constructions with a countable family of non-cumulative universe levels, similar
to the type theory of Lean; $\lambda_\to^*$ is simply typed lambda calculus with a countable
family of universe levels, without $\mathsf{U}_0$ and without the typing relation
$\mathsf{U}_\ell : \mathsf{U}_{\ell + 1}$, intended to model the output of Lean-auto's
monomorphization; $\lambda_\to$ is simply typed lambda calculus. All three systems will be
specified as pure-type systems $(\mathcal{S}, \mathcal{A}, \mathcal{R})$, where $\mathcal{S}$
is the set of sorts, $\mathcal{A} \subseteq \mathcal{S}^2$ is the set of axioms, and
$\mathcal{R} \subseteq \mathcal{S}^3$ corresponds to the formation rules.

\begin{definition} $\lambda C$ is the pure type system $(\mathcal{S}, \mathcal{A}, \mathcal{R})$ where
  $$\mathcal{S} := \{\mathsf{U}_\ell | \ell \in \mathbb{N}\} \ \ \ \mathcal{A} := \{(\mathsf{U}_\ell, \mathsf{U}_{\ell + 1}) | \ell \in \mathbb{N}\}$$
  $$\mathcal{R} := \{(\mathsf{U}_\ell, \mathsf{U}_m, \mathsf{U}_{\mathsf{imax}(\ell, m)}) | \ell \in \mathbb{N}, m \in \mathbb{N}\}$$
  $$\mathsf{imax}(m, n) := \left\{\begin{aligned}
    \mathsf{max}(m, n), & & n > 0 \\
    0, & & n = 0
  \end{aligned}\right.$$  
\end{definition}

\begin{definition} $\lambda_\to^*$ is the pure type system $(\mathcal{S}, \mathcal{A}, \mathcal{R})$ where
  $$\mathcal{S} := \{\mathsf{U}_\ell | \ell \in \mathbb{N}^*\} \cup \{\mathsf{U}_\ell' | \ell \in \mathbb{N}^*\} \ \ \
    \mathcal{A} := \{(\mathsf{U}_\ell, \mathsf{U}_\ell') | \ell \in \mathbb{N}^*\}$$
  $$\mathcal{R} := \{(\mathsf{U}_\ell, \mathsf{U}_m, \mathsf{U}_{\mathsf{max} \{l, m\}}) | \ell \in \mathbb{N}^*, m \in \mathbb{N}^*\}$$
\end{definition}

\begin{definition} $\lambda_\to$ is the pure type system $(\mathcal{S}, \mathcal{A}, \mathcal{R})$ where
  $$\mathcal{S} := \{\mathsf{U}_1, \mathsf{U}_1'\} \ \ \ \mathcal{A} := \{(\mathsf{U}_1, \mathsf{U}_1')\} \ \ \ 
    \mathcal{R} := \{\mathsf{U}_1, \mathsf{U}_1, \mathsf{U}_1\}$$
  This is equivalent to simply typed lambda calculus, where $\mathsf{U}_1$ and $\mathsf{U}_1'$ are
  usually denoted as $*$ and $\square$, respectively.
\end{definition}

\begin{theorem} A $\lambda_\to^*$ problem is provable iff it is $\lambda_\to$ provable after forgetting
  universe levels, assuming the existence of lifting functions in $\lambda C$.
\end{theorem}
\begin{proof} \textbf{TODO}
  (Put this after the definition of provability)
  (Maybe a direct induction would be more elegant?)
  If a $\lambda_\to^*$ problem is provable, obviously it is $\lambda_\to$ provable
  after forgetting the universe levels. If a $\lambda_\to^*$ problem is $\lambda_\to$ provable
  after forgetting the universe levels, note that the lifting of a $\lambda_\to^*$ problem is sort of
  an instance of the problem after forgetting universe levels, hence provable. Moreover,
  a $\lambda_\to^*$ problem is provable iff its lifting is provable.  
\end{proof}

\subsection{Essentially higher-order problem}

\begin{definition} Let $\sigma : V \to \mathcal{T}$ be a mapping.
  Define its extension $\overline{\sigma} : \mathcal{T} \to \mathcal{T}$ as
  $$\overline{\sigma}(\mathsf{U}_\ell) := \mathsf{U}_\ell$$
  $$\overline{\sigma}(x) := \sigma(x), \text{ for }x \in V$$
  $$\overline{\sigma}(M \ N) := \overline{\sigma}(M) \ \overline{\sigma}(M)$$
  $$\overline{\sigma}(\lambda x : s. M) := \lambda x : \overline{\sigma}(s). \overline{\sigma[x \mapsto x]}(M)$$
  $$\overline{\sigma}(\forall x : s. M) := \forall x : \overline{\sigma}(s). \overline{\sigma[x \mapsto x]}(M)$$
  where
  $$\sigma[u \to t](x) := \left\{\begin{aligned}
    & t & & x = u \\
    & \sigma(x) & & x \in V \backslash \{u\}
  \end{aligned}\right.$$
\end{definition}

\begin{definition} A substitution is a triple $(\Gamma, \Gamma', \sigma)$ where $\Gamma, \Gamma'$ are contexts
  and $\sigma : V \to \mathcal{T}$, such that for all $(u : \tau) \in \Gamma$,
  $$\Gamma' \vdash \sigma(u) : \overline{\sigma}(\tau)$$
  $\Gamma$ is called the domain of the substitution, and $\Gamma'$ is called the codomain of the substitution.
\end{definition}

\begin{theorem}
  Let $(\Gamma, \Gamma', \sigma)$ be a substitution. If $\Gamma \vdash t : s$, then $\Gamma' \vdash \overline{\sigma}(t) : \overline{\sigma}(s)$
\end{theorem}
\begin{proof} Induction on the derivation of $\Gamma \vdash t : s$, note that $\Gamma'$ and $\sigma$
  should be universally quantified in the induction hypothesis.
\end{proof}

\begin{definition} Many-sorted higher-order logic (HOL) is defined as $\lambda_\to^*$ augmented
  with the following symbols:
  \begin{enumerate}
    \item $\mathsf{Bool}$
    \item $\bot'$ and $\to'$
    \item $\forall'_t$, for each $t \in \mathcal{T}$
  \end{enumerate}
  
  \noindent and the following derivation rules:
  $$\frac{}{\vdash \mathsf{Bool} : \mathsf{U}_1} \ \ \ \ \frac{}{\Gamma \vdash \bot' : \mathsf{Bool}}$$
  $$\frac{}{\Gamma \vdash \to' : \mathsf{Bool} \to \mathsf{Bool} \to \mathsf{Bool}} \ \ \ \
  \frac{\Gamma \vdash s : \mathsf{U}_\ell}{\Gamma \vdash \forall'_s : (s \to \mathsf{Bool}) \to \mathsf{Bool}}$$
  
  \noindent For simplicity, we use $\forall' (x : \alpha), t$ as a shorthand for $\forall'_\alpha \ (\lambda x : \alpha. t)$

  \noindent The canonical embedding $\pi : \mathcal{T}_{\text{HOL}} \to \mathcal{T}$ of HOL into $\lambda C$ is defined as follows:
  $$\pi(\mathsf{Bool}) := \mathsf{U}_0 \ \ \ \pi(\mathsf{U}_\ell) := \mathsf{U}_\ell \ \ \
    \pi(\mathsf{U}_\ell') := \mathsf{U}_{\ell + 1}$$
  $$\pi(\bot') := \forall (\alpha : \mathsf{U}_0). \alpha \ \ \
  \pi(\to') := \lambda (p \ q : \mathsf{U}_0). \forall (x : p). q$$
  $$\pi(\forall_t') := \lambda (p : \pi(t) \to \mathsf{U}_0). \forall (x : \pi(t)). p \ x$$
  $$\pi(x) := x, \text{ for } x \in V$$
  $$\pi(M \ N) := \pi(M) \ \pi(N) \ \ \ \pi(\lambda x : s. M) := \lambda x : \pi(s). \pi(M)$$

  \noindent $\pi$ is extended to contexts via the following definition:
  $$\pi(\emptyset) := \emptyset \ \ \ \pi(\Gamma, x : \sigma) := \pi(\Gamma), x : \pi(\sigma)$$

  \noindent \textbf{TODO: Is this really equivalent to higher-order logic?}

\end{definition}

\begin{theorem}\label{ceptj} Canonical embedding preserves judgement, i.e. if $\Gamma \vdash t : s$ in HOL, then
  $\pi(\Gamma) \vdash \pi(t) : \pi(s)$ in $\lambda C$ \end{theorem}
\begin{proof} Induction on the derivation rules of HOL. \end{proof}

\begin{definition} An (HOL/$\lambda C$) problem is a tuple $(\Gamma, p)$, denoted
  as $\Gamma \vdash? p$, where $\Gamma$ is a (HOL/$\lambda C$)
  context, called the hypotheses of the problem, and $p$ is an
  (HOL/$\lambda C$) term, called the goal of the problem. A $\lambda C$ problem
  $\Gamma \vdash? p$ is provable iff there exists a $\lambda C$ term $t$ such that
  $\Gamma \vdash t : p$. An HOL problem $\Gamma \vdash? p$ is provable iff there exists
  a $\lambda C$ term $t$ such that $\pi(\Gamma) \vdash t : \pi(p)$.
\end{definition}

\begin{definition} A $\lambda C$ problem $\Gamma \vdash? p$ is essentially higher-order provable (EHOP)
  iff there exists a provable HOL problem $\Gamma' \vdash? p'$ and a substitution
  $(\pi(\Gamma'), \Gamma, \sigma)$ such that $p = \overline{\sigma}(\pi(p'))$.
\end{definition}

\begin{theorem}
  If a $\lambda C$ problem $\Gamma \vdash? p$ is EHOP, then it is provable.
\end{theorem}
\begin{proof} By the definition of EHOP, there exists a provable HOL problem
  $\Gamma' \vdash? p'$ and substitution $(\pi(\Gamma'), \Gamma, \sigma)$ such that
  $p = \overline{\sigma}(\pi(p'))$. By the definition of HOL provability, there exists
  a term $t'$ such that $\pi(\Gamma') \vdash t' : \pi(p')$. By theorem \ref{ceptj},
  $\Gamma \vdash \overline{\sigma}(t') : \overline{\sigma}(\pi(p'))$, thus $\Gamma \vdash? p$
  is provable.
\end{proof}

\noindent We define logical symbols in $\lambda C$ as follows:
$$\bot := \forall p : \mathsf{U}_0. p \ \ \ (\neg) := \lambda p : \mathsf{U}_0. p \to \bot$$
$$(\land) := \lambda p \ q : \mathsf{U}_0. \forall r : \mathsf{U}_0. (p \to q \to r) \to r$$
$$(\lor) := \lambda p \ q : \mathsf{U}_0. \forall r : \mathsf{U}_0. (p \to r) \to (q \to r) \to r$$
$$(\leftrightarrow) := \lambda p \ q. (p \to q) \land (q \to p)$$
$$(=_\ell) := \lambda \alpha : \mathsf{U}_\ell. \lambda x \ y : \alpha. \forall p : \alpha \to \mathsf{U}_0. (p \ x \leftrightarrow p \ y)$$
$$(\exists_\ell) := \lambda \alpha : \mathsf{U}_\ell. \lambda p : \alpha \to \mathsf{U}_0. \forall q : \mathsf{U}_0. ((\forall x : \alpha. p \ x \to q) \to q)$$

\noindent The symbols $\neg', \land', \lor', \leftrightarrow'$ are defined in HOL in the
same way, except that the $\mathsf{U}_0$s are replaced with $\mathsf{Bool}$ and the $\to$s are
replaced with $\to'$. Equality and existential quantifier in HOL are defined as follows:
$$(=_s') := \lambda x \ y : s. \forall' p : \alpha \to \mathsf{Bool}. (p \ x \leftrightarrow' p \ y)$$
$$(\exists_s') := \lambda p : \alpha \to \mathsf{Bool}. \forall' q : \mathsf{Bool}. ((\forall' x : \alpha. p \ x \to' q) \to' q)$$

\noindent We also assume that excluded middle, i.e. $\mathsf{em} : \forall p : \mathsf{U}_0, p \lor \neg p$,
is implicitly contained in the hypotheses of all $\lambda C$ problems. Similarly, $\mathsf{em}' : \forall p : \mathsf{Bool}, p \lor \neg p$
is assumed to be implicitly contained in the hypotheses of all HOL problems.

\begin{example} Consider the $\lambda C$ problem $\Gamma \vdash? p$ where
\begin{align*}
  \Gamma := \ & \mathbb{N} : \mathsf{U}_1, \mathsf{Fin} : \mathbb{N} \to \mathsf{U}_1,
  \mathsf{add} : \forall n : \mathbb{N}, (\mathsf{Fin} \ n \to \mathsf{Fin} \ n \to \mathsf{Fin} \ n), n : \mathbb{N} \\
  p := \ & (\forall (u \ v : \mathsf{Fin} \ n). \mathsf{add} \ n \ u \ v = \mathsf{add} \ n \ v \ u) \to \\
  & \ \ \ \forall (u \ v \ w : \mathsf{Fin} \ n). \mathsf{add} \ n \ (\mathsf{add} \ n \ x \ y) \ z = \mathsf{add} \ n \ z \ (\mathsf{add} \ n \ y \ x)
\end{align*}
Given
\begin{align*}
  \Gamma' := \ & \alpha : \mathsf{U}_1, f : \alpha \to \alpha \to \alpha \\
  p' := \ & (\forall' (u \ v : \alpha). f \ u \ v =_\alpha' f \ v \ u) \to' \\
  & \ \ \ \forall' (u \ v \ w : \alpha), f \ (f \ u \ v) \ w =_\alpha' f \ w \ (f \ v \ u)
\end{align*}
The higher-order problem $\Gamma' \vdash? p'$ is provable. Moreover, given
$$\sigma(\alpha) := \mathsf{Fin} \ n, \sigma(f) := \mathsf{add} \ n$$
The triple $(\pi(\Gamma'), \Gamma, \sigma)$ forms a substitution, and $p = \overline{\sigma}(\pi(p'))$.
Therefore, $\Gamma \vdash? p$ is EHOP.
\end{example}

\noindent Note that moving implications in the goal into hypotheses (and vice versa) may
change the EHOP status of a problem. For example,
$$\alpha : \mathsf{U}_1, x : \alpha, p : \alpha \to \mathsf{U}_0 \vdash? p \ x \to p \ x$$
is EHOP. However, if we introduce $p \ x$ into the hypotheses, the problem is no longer EHOP:
\begin{equation}\label{hypnehop}
  \alpha : \mathsf{U}_1, x : \alpha, p : \alpha \to \mathsf{U}_0, h : p \ x \vdash? p \ x
\end{equation}

\begin{theorem}
  The $\lambda C$ problem \eqref{hypnehop} is provable but not EHOP.
\end{theorem}
\begin{proof}
  Note that $h : p \ x$ under the hypotheses of \eqref{hypnehop}, thus \eqref{hypnehop} is provable.
  To show that \eqref{hypnehop} is not EHOP, we use proof by contradiction. Suppose there
  is an HOL problem $\Gamma' \vdash p'$ and a substitution $(\Gamma', \Gamma, \sigma)$ such
  that $p = \overline{\sigma}(\pi(p'))$. Then, the $\beta$-nf of $p'$ must be of the form
  $f \ t_1 \ \dots \ t_k$ where $f$ is a variable. Note that $\Gamma'$, as a context of $\lambda_\to^*$,
  consists solely of HOL (type or term) variable declarations. \textbf{TODO:} Use model theory
\end{proof}

\input{example}

\begin{credits}
\subsubsection{\ackname} A bold run-in heading in small font size at the end of the paper is
used for general acknowledgments, for example: This study was funded
by X (grant number Y).

\subsubsection{\discintname}
It is now necessary to declare any competing interests or to specifically
state that the authors have no competing interests. Please place the
statement with a bold run-in heading in small font size beneath the
(optional) acknowledgments\footnote{If EquinOCS, our proceedings submission
system, is used, then the disclaimer can be provided directly in the system.},
for example: The authors have no competing interests to declare that are
relevant to the content of this article. Or: Author A has received research
grants from Company W. Author B has received a speaker honorarium from
Company X and owns stock in Company Y. Author C is a member of committee Z.
\end{credits}
%
% ---- Bibliography ----
%
% BibTeX users should specify bibliography style 'splncs04'.
% References will then be sorted and formatted in the correct style.
%
% \bibliographystyle{splncs04}
% \bibliography{mybibliography}
%
\begin{thebibliography}{8}
\bibitem{ref_article1}
Author, F.: Article title. Journal \textbf{2}(5), 99--110 (2016)

\bibitem{ref_lncs1}
Author, F., Author, S.: Title of a proceedings paper. In: Editor,
F., Editor, S. (eds.) CONFERENCE 2016, LNCS, vol. 9999, pp. 1--13.
Springer, Heidelberg (2016). \doi{10.10007/1234567890}

\bibitem{ref_book1}
Author, F., Author, S., Author, T.: Book title. 2nd edn. Publisher,
Location (1999)

\bibitem{ref_proc1}
Author, A.-B.: Contribution title. In: 9th International Proceedings
on Proceedings, pp. 1--2. Publisher, Location (2010)

\bibitem{ref_url1}
LNCS Homepage, \url{http://www.springer.com/lncs}, last accessed 2023/10/25
\end{thebibliography}
\end{document}
